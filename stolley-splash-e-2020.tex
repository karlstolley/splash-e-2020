\documentclass[sigplan,screen]{acmart}

\usepackage[utf8]{inputenc}
\usepackage{microtype}

\AtBeginDocument{%
  \providecommand\BibTeX{{%
    \normalfont B\kern-0.5em{\scshape i\kern-0.25em b}\kern-0.8em\TeX}}}

\setcopyright{acmcopyright}
\acmPrice{}
\acmDOI{10.1145/3426431.3428656}
\acmYear{2020}
\copyrightyear{2020}
\acmSubmissionID{splashe20main-id5-p}
\acmISBN{978-1-4503-8180-2/20/11}
\acmConference[SPLASH-E '20]{Proceedings of the 2020 ACM SIGPLAN SPLASH-E Symposium}{November 20, 2020}{Virtual, USA}
\acmBooktitle{Proceedings of the 2020 ACM SIGPLAN SPLASH-E Symposium (SPLASH-E '20), November 20, 2020, Virtual, USA}



\begin{document}
\title[Objective Typography]{CSS Instruction Enhanced by Objective Typography}

\author{Karl Stolley}
\email{kstolley@iit.edu}
\orcid{1234-5678-9012}
\affiliation{
  \institution{Illinois Institute of Technology}
  \streetaddress{Institution Street Address}
  \city{Chicago}
  \state{IL}
  \country{USA}
  \postcode{60616}}

\begin{abstract}
This course experience report details the teaching of Cascading Style Sheets (CSS) constrained by the rules of objective typography. The approach guides students in applying those rules to a subset of fewer than a dozen CSS properties. Students learn how to determine and reason about rule-governed values and ratios according to typographic principles. When successful, students produce typeset text that is accessible and readable across the range of screens and user-set preferences on web-enabled devices. Students learn to visually and mathematically verify the execution of their designs, and to apply the rules of objective typography to other areas of CSS, such as grid-based page layout. Experiential evidence suggests that these techniques do transfer to other aspects of CSS, but formal study is needed.
\end{abstract}

\begin{CCSXML}
<ccs2012>
<concept>
  <concept_id>10003456.10003457.10003527.10003531.10003535</concept_id>
  <concept_desc>Social and professional topics~Information technology education</concept_desc>
  <concept_significance>500</concept_significance>
</concept>
<concept>
  <concept_id>10003456.10003457.10003527.10003528</concept_id>
  <concept_desc>Social and professional topics~Computational thinking</concept_desc>
  <concept_significance>300</concept_significance>
</concept>
</ccs2012>
\end{CCSXML}

\ccsdesc[500]{Social and professional topics~Information technology education}
\ccsdesc[300]{Social and professional topics~Computational thinking}

\keywords{Cascading Style Sheets, typography, web design, curriculum}


\maketitle

\section{Introduction}

At least two factors make CSS a challenging language to teach. First, CSS bears little resemblance to other languages in the web curriculum or in the curriculums of computer science and information technology generally. Even comparing languages only in the web curriculum, HTML can trace its ancestry to SGML and its contemporaries to XML. JavaScript has a complicated history but shares the features of many prototyped, procedural languages. By contrast, CSS’s closest relative is an obscure one: Simple Tree Transformation Sheets (STTS)\cite{w3c:briefhistory}, which used the selector-declaration syntax found in CSS. CSS's outlier status means that knowledge of other languages has limited transfer to CSS.

Second, CSS is a challenge to teach because the language continues to grow. Since the original CSS specification was published in late 1996, the CSS language has expanded considerably in capability as well as complexity: the 53 properties in the CSS 1 specification more than doubled to 115 in CSS 2.1/2.2. The modules that comprise CSS3 included  363 properties \cite{jom:css}. As of October 2020, the World Wide Web Consortium (W3C) estimates that CSS contains or is proposed to contain 533 distinct properties, spread across 145 different technical reports and editors' drafts \cite{w3c:iop}. At its most basic, CSS works by targeting and styling specific structures found in HTML documents. Because each CSS property has a very specific purpose (specifying text and background colors; italicizing or bolding text, for basic examples), CSS properties are not generalizable or easily used to suit other purposes. That stands in contrast to the core methods and functions of most other languages. Instruction in CSS can therefore devolve into trying to cover a growing list of properties.

 Students also face their own challenges learning CSS. As the language has grown, overwhelmed students often skip the fundamentals in favor of searching for tutorials that describe complicated visual effects and gimmicks, such as parallax scrolling, which inevitably lead to a lot of cut-and-paste code and little transferable learning.

CSS instruction constrained by typographic principles addresses the  complexity of CSS and attempts to solve the following problems students routinely face learning CSS:


\begin{itemize}
  \item \textbf{Where to start?} CSS can specify many visual qualities,  in just about any order. CSS neither enforces an order nor establishes any given properties as necessary, in contrast to the precise requirements of HTML.
  \item \textbf{What values to use?} And what units to express CSS values in? Left to their own devices, students pick values more or less out of thin air, or from other, inappropriate contexts: {\itshape Google Docs defaults to a text size of 11, so I’ll use 11 on this web page, too.}
  \item \textbf{When is it “finished”?} Compared to a set of style declarations in CSS, the content-completeness of an HTML document or the feature-completeness of a JavaScript function are more obviously verified, even by beginners.
\end{itemize}

The approach to CSS instruction described here developed out of an effort to present CSS as equally systematic and rule-governed as well written HTML and JavaScript. Students are introduced to CSS at a critical moment, when they have become more confident in their HTML skills but are eagerly looking to create their own custom page designs.


\section{Instructional Context}

This approach has been used several times over a four-year period in the first course of a two-course sequence in introductory web-development. Both courses in the sequence are required of all undergraduate majors in an information technology and management program. Students often take the sequence in their sophomore years. The first course, Fundamentals of Web Development, covers the syntax and semantics of standards-based HTML, CSS, and JavaScript. The second course, Human-Computer Interaction and Web Design, introduces theories of user-centered design and human-computer interaction, anchored to common web design problems.

The Fundamentals of Web Development course typically enrolls between 30 and 40 students. Classroom instruction is largely lecture based, featuring live-coded demonstrations with ample opportunity for student questions. Students complete between five and ten small lab assignments, including one on objective typography and CSS, which are discussed on the course's electronic discussion boards. 

The bulk of student work in the course, however, is organized around three major projects, completed by each student individually. The first project requires students to develop the semantic HTML foundation for a multi-page website. Students are free to choose the subject matter of their sites, which has lead in almost all cases to students being substantially invested in all of the major course projects. Most students opt to begin developing a professional web presence for themselves, while others choose to create websites for campus clubs or even family businesses. The second project adds CSS and audio-visual elements to the HTML work completed in the first project, although students continue to refine and improve their HTML. The third and final project is a revision and refactoring of the second project, in response to multiple rounds of peer and instructor feedback, and requires students to include a JavaScript feature. A copy of the complete course syllabus is available on the web\cite{kas:fwd}.

Both courses in the sequence teach accessibility as a core web-engineering principle. That is expressed in part by the philosophy of progressive enhancement, which the course projects reflect. Progressive enhancement rests on a foundation of semantic HTML, free from all presentational elements and attributes. The HTML is then styled by a single, site-wide CSS file. HTML and CSS can be further augmented by unobtrusive JavaScript that leverages event listeners and the browser’s Document Object Model (DOM) to add page interactivity and feature enhancements not achievable by HTML or CSS.

\subsection{Objective Typography Principles}

Typography is a familiar design feature for students to approach. Students may not have worked with color theory or page layout before, but all have selected a preferred typeface on a document or in a slide template. Typography also functions at a human scale, making it a useful constant when designing for the web across the range of wearable, handheld, and desktop devices. Graphic designer Ellen Lupton observes that “Words originated as gestures of the body. The first typefaces were modeled on the forms of calligraphy.” \cite[p.~13]{el:type}.

Objective rules for the use of type--“the correct spaces between letters and words and the length and spacing of lines conducive to easy reading” \cite[p.~19]{mb:grid}--trace their origins to the at least the Industrial Revolution. They achieve fuller and more exacting description in the work of mid-twentieth-century European designers such as Josef Müller-Brockmann’s {\itshape Grid Systems in Graphic Design}, which specifically labeled the principles as constituting “objective typography” \cite[p.~7]{mb:grid}.

The basic rules of objective typography are echoed and refined across a number of contemporary book-length works that I have assigned in whole or as excerpts in web development classes: Ellen Lupton’s {\itshape Thinking with Type}, Robert Bringhurst’s {\itshape The Elements of Typographic Style}, and Jason Santa Maria’s {\itshape On Web Typography} \cite{el:type,rb:style,jsm:owt}. An open-access book that has also been a student favorite is Matthew Butterick’s {\itshape Practical Typography} \cite{mb:pt}. It is essential for students to have some available typography reference to understand both the significance of the work that they are doing, and also to consistently apply and express the rules of objective typography through CSS.

\subsection{Instructional Approach in Brief}

Applying the rules of objective typography to CSS focuses students on determining two values--a font-size and a line-height--and initially applying those values across fewer than a dozen CSS properties. The font-size is chosen based upon a narrow range of readable values at a given viewport or device size: too small, and the text is difficult to perceive, let alone read. Too large, and on phone-sized devices and screens, too few words appear on a single line, which hampers reading by making writing appear choppy. The line-height, which is selected based on the font-size for comfortable reading, establishes a predictable, repeating baseline for text to sit upon. That repeating baseline is also known as the typographic grid. 

Of course, text set at a single size on a uniform grid is neither exciting nor pleasant to read, so students learn to apply a ratio to their base font-size to arrive at precise values for larger text (headings, for example) and smaller text (captions and fine print). Those alternative type sizes are, however, always anchored by the original readable, accessible base font-size. Students also learn to add additional space above and below headings, list items, and other text features. That additional space is always a fraction or multiple of the base line-height value.

\begin{figure}
  \includegraphics[width=\linewidth]{rdv}
  \caption{The responsive design view (RDV) in Mozilla Firefox.}
  \Description{A screenshot showing a page of HTML in a browser, with reset styles applied.}
  \label{fig:rdv}
\end{figure}


\section{Instructional Rationale and Method}

Applied to CSS instruction, objective typography offers a number of advantages:

\begin{itemize}
  \item It is rule- and principle-governed, which clarifies the objectives of both instruction and assessment
  \item It limits the scope of initial CSS instruction to just a handful of properties and techniques
  \item Its expression in CSS exposes students to numerous features of the language, especially inheritance, where a value set on one property affects the initial value or behavior of another property, even when set on another HTML element
  \item It introduces students to the process of translating and transposing external rules or visual plans from abstract concepts to concrete declarations in CSS
  \item It establishes a core set of values and units that can be carried over into advanced layout methods and grid-based design
\end{itemize}

Teaching CSS constrained by objective typography is systematic but leaves open room for experimentation: Students work with both their own written copy and dummy {\itshape Lorem ipsum} text to determine a reasonable size for the body copy, initially expressed in absolute, pixels units. [Something about the `html` selector.] From there, students experiment with different values on the \verb|line-height| property to find an appropriate leading (pron. {\itshape ledding}) to separate the lines of running text. That value becomes the fundamental base-line grid for the page. Block elements (headings, paragraphs, lists and list items) can then be offset from one another with additional vertical space, based on the \verb|line-height| value. Students then convert those absolute values into relative values for greater accessibility.

[Combine with previous para] For the purposes of learning this technique, students can be provided with a semantically structured HTML document from the instructor, or they can work from their own HTML, perhaps as part of a larger project. The HTML should use the viewport \verb|<meta>| element to set mobile devices to display at their native resolutions.

Because of the varying availability of typefaces on different operating systems, instructors should suggest a few web-available typefaces, such as from Google Fonts, for students to work with. Ideally, the candidate typefaces should have anatomical features that lend themselves to long-form reading: moderate stroke contrast, a large x-height, and pronounced ascenders and descenders \cite[p.~36]{jsm:owt}. The typefaces in Figure \ref{fig:bvos} all exhibit those features.

\subsection{Viewports and Devices}


Finally, students begin to open up the RDV to larger viewports and make targeted adjustments to their stylesheets, often using CSS media queries, to ensure an optimal text setting across a full range of device and viewport sizes.

after applying a reset stylesheet, which effectively strips away all default browser styles \cite{em:rc}, students are shown how to work with a mobile-sized viewport. Students begin by working with the responsive design view (RDV) in a development browser such as Mozilla Firefox Developer Edition (Figure \ref{fig:rdv}). The RDV provides a precise readout of the viewport's scale, and prevents students from the common pitfall of always designing for the web at the maximum viewport size their computer can display. The viewport in the RDV should be collapsed to a size that approximates the dimensions of a mobile phone. Students need to be reminded of this often. Students should be shown how and encouraged to make use of a lightweight local web server, such as the \verb|http-server| package written for Node.js \cite{npm:http}, so that they can learn as early as possible to view projects in progress over the local network on actual mobile and tablet devices, not just their own chosen development browser. It is usually only when students look at their work on a phone that they begin to really grasp designing for the smallest screens first.

\subsection{Sizing Body Type}

Selecting a readable size for body type is a critical step in the typesetting process, because that size value determines all of the other values that follow. While keeping critical eyes on their browser's RDV and their mobile phones, students experiment with discovering a suitable base \verb|font-size| value for the page.

Students are typically surprised to learn that there isn’t a precise, universally readable size value. That is due in part to how typefaces are drawn: the optical size of a typeface (that is, its visible letterforms) differs from its body size (the invisible boxes around letterforms, which is what the \verb|font-size| property sets). That means that two typefaces set to the same exact \verb|font-size| value may nevertheless render at very different optical sizes [figure?]. For that reason, in lab assignments I will often require a specific typeface for students to use. However, for their major projects, students are responsible for choosing their own typefaces.

Regardless of the typeface to be sized, students are generally more confident and successful in this work when introduced to some additional constraints for sizing text, including common browser defaults and accessibility guidelines. By default, most desktop browsers set their base font-size to 16 pixels, which the latest version of the Web Content Accessibility Guidelines (WCAG) identifies as “a reasonable size to assume” \cite{w3c:wcag}. On the larger end of a starting range of sizes, the Lighthouse Guild, formerly the National Association for the Visually Handicapped, recommends a size range of 16 to 18 points while also noting the optical-size differences between typefaces \cite{lhg:mtl}. At approximately 1.333 CSS pixels per point, a workable range of values for the base font size runs from 16 to 24 pixels.

I often find it useful at this point and throughout the process to remind students that pixel values are only temporary convenience values for use at the design stage, when sizing is still in flux. Eventually, as detailed in section 3.6 below, all pixel units will be converted to relative \verb|em| or \verb|rem| units, and therefore made responsive to users who set their browser's base font size to something other than the 16-pixel default.

\subsection{Establishing the Baseline Grid}

This is often the step, however, where students encounter a great deal of conceptual difficulty. Their experiences setting line-spacing in word processors, for example, seems to have trained them to think in terms of single- and double-spacing. The concept of setting a value on the CSS \verb|line-height| property in the same pixel units to produce a grid of baselines is very different, and often requires repeated classroom demonstrations in addition to specific feedback about the line-height value students choose on labs and projects.

Guiding principles again help students arrive at their baseline-grid value. Typographic convention sets the baseline grid anywhere from 120\% to 180\% of the base font size for body text \cite[p.~92]{jsm:owt}. The ideal baseline grid for a page is determined both by the size of the body text as well as the width of the column. The number of words on a line of text in a column is known as the {\itshape measure}; a readable measure on a single column of text is often between 45 and 75 characters, including spaces \cite[pp. 26–27]{rb:style}. Smaller measures can be set closer than larger measures, so long as the space is doing its primary job: aiding the eye's movement comfortably from the end of one line of text to the beginning of the next.

In addition to those principles, students also benefit greatly from the aid of a diagnostic grid overlay such as can be provided by Basehold.it or using a simple piece of instructor-provided CSS and SVG \cite{basehold}. With lines showing the size of their chosen baseline grid, students can better visually confirm the accuracy of their baseline grid and how their type sits within it.

I encourage students to choose values for the \verb|line-height| property that divide evenly into 2 and 4 for expressing half- and quarter-line adjustments between elements, as is described below in section 3.5. Students also must learn to switch their grid overlay on and off, because the presence of the diagnostic lines can make it more difficult for students to assess when they have arrived at a line-height value that is most comfortable for reading.

\subsection{Sizing Accent Type}

\begin{figure}
  \includegraphics[width=\linewidth]{rdv-narrow}
  \caption{Text precisely set on a repeating baseline grid, with diagnostic grid lines shown. The larger headline text occupies multiple grid lines, while each line of paragraph copy occupies just a single line.}
  \label{fig:rdv-narrow}
\end{figure}

Up until this point, students have only been setting properties on the \verb|html| selector. With a base font-size and a repeating baseline grid value in hand, students are ready to make size adjustments to the text on their page, most often beginning with headings.

Without being offered guidance, student will naturally want to write arbitrarily larger sizes on the heading selectors in their stylesheets. A better, rule-governed approach is to encourage students to experiment with different mathematical ratios to determine a modular scale, also known as a typographic scale, for precisely incremented font sizes. Students report enjoying exploring modularscale.com and its simple controls for specifying a base font-size as well as one of a number of different ratios for resizing text \cite{modscale}.

A 4:3 ratio, for example, with a base font-size of 18px would produce a typographic scale with the sizes 18px, 23.994px, 31.984px, and so on as shown in Figure \ref{fig:modscale}. Advanced students in particular can be confused by partial pixels, like 31.984, as only whole pixels can be lit. But it serves a good instructional reminder that pixel values are convenience values, which will later be converted to relative units. The browser will handle the rounding of either unit to whole-pixel values.

To keep the base-line grid consistently sized relative to larger text, students need to adjust the \verb|line-height| property on text that is larger or substantially smaller than their base font size, as shown in Figure \ref{fig:rdv-narrow}. Assuming a base font-size of 18px and a baseline grid of 22px, a heading set to 23.994px could be set comfortably on a gridline and a half (33px) or on a fully doubled gridline (44px). As Figure \ref{fig:rdv-narrow} shows, the paragraph text continues to adhere to the grid, indicating that the gridline adjustment on the heading was correctly set.

\begin{figure}
  \includegraphics[width=\linewidth]{modular-scale}
  \caption{A 4:3 modular scale with an 18-pixel base font-size as shown on modularscale.com.}
  \label{fig:modscale}
\end{figure}


\subsection{Refining the Grid}

Having established a base font size, a baseline grid, and a typographic scale, students move on to adding grid-determined space around elements. Both the \verb|margin| and \verb|padding| properties in CSS are appropriate for this task, although \verb|padding| behaves more predictably than \verb|margin|, as CSS will collapse adjacent margin values under particular circumstances.

An entire line of empty space can follow paragraph elements, as shown in Figure \ref{fig:rdv-narrow}, while list items can be separated by a half line of space. An additional half line of space set on the list itself ensures that a list will be followed by an entire line of empty space, just like paragraphs.

Additional space, derived from the \verb|line-height| value for the typographic grid, can be added to the \verb|html| selector on its \verb|padding| property.

\subsection{Relative Units and Responsiveness}

\begin{figure}
  \includegraphics[width=\linewidth]{rdv-wide}
  \caption{Text set larger in a wider RDV viewport, with diagnostic grid lines shown.}
  \Description{At a larger viewport, the text is run larger. Gridlines are still constant, and size relationships between headings and boy copy remain. Additional space appears on either side of the single column of text.}
  \label{fig:rdv-wide}
\end{figure}

With a diagnostic grid overlay still displaying, students begin to gradually convert their absolute pixel-values to relative em-values. Because browsers default to a 16-pixel em, students convert their base \verb|font-size| by dividing by 16. So a 19-pixel \verb|font-size| is equivalent to \verb|1.1875em|.

That value becomes the new em value for the page. If the 19-pixel text is set on a 24-pixel baseline, the relative \verb|line-height| value is arrived at by calculating 24 ÷ 19, or \verb|1.2631578947em|. With each conversion to a relative unit, students should refresh their browsers and ensure their work still shows perfect alignment with the overlay . Any deviations from the grid means that a relative value was calculated incorrectly.

This is another pain-point for some students, who race through their stylesheets and converting to em values without checking their work in the browser. In that situation, the student has to return to the stylesheet and do a great deal of value-checking to determine the source of the errors. I encourage students to always preserve their unit-conversions in CSS comments, which helps them to track down errors as well as keeping clear for themselves how their relative values were calculated.

With relative units accurately in place, students can then begin to open up the RDV viewport further. Once the lines of text become too long to comfortably read, or exceed the recommended lengths for line measures referenced earlier, and adjustment is needed. The primary mechanism for making adjustments are via CSS media queries, which conditionally apply their contained CSS based on a screen condition, such as \verb|min-width|. Students should use the width as displayed in the RDV to write the query. A media query may need only to introduce a larger \verb|font-size| set on the \verb|html| selector. All of the carefully determined ratios and size relationships will scale accordingly. Additional space to the left and right sides of the text column can also help keep the measure of the lines readable, as shown in Figure \ref{fig:rdv-wide}.

 Additionally, WCAG Success Criterion 1.4.4 calls for users to be able to resize text up to 200\% “without loss of content or functionality” \cite{w3c:wcag}, meaning that students can experiment with smaller sizes, provided that those sizes work when doubled: either by adjusting the value the stylesheet or using the zoom feature in their development browser.

\section{Conclusion}

This approach to CSS instruction has proved challenging but rewarding for students. Instead of focusing on a growing number of properties, students learn to take their time arriving at very precise values for a select number of properties. The unique combination of typeface, font size, and line height makes cut-and-paste approaches to CSS unworkable. Students take ownership over their work, and not uncommonly express a great deal of pride when their work scales up with user-set font preferences in the browser. 

Anecdotally, having employed this method, students generally seem much more adept at commanding additional complex CSS modules later in the course, particularly CSS Flexible Boxes and CSS Grid, which can be set to use the base line-height value from the student's initial typesetting.

Note that applying the fundamental principles of objective typography does not come close to fully exhausting the possibilities of CSS typography: the basic instructional approach excludes typographic flourishes and stylistic enhancements, especially font variants such as small caps, that can now be represented in stunning clarity on high-density displays.

\subsection{Planned Enhancements}

During fall semester 2020, this approach will be enhanced further to include additional responsive typography features unique to CSS, including molten leading and CSS locks \cite{tb:ml,tb:cl}.

Simultaneously, so as to gather data about the effectiveness of the technique, an instrument will be developed and tested with volunteer enrolled students for further refinement in the future.

\bibliographystyle{ACM-Reference-Format}
\bibliography{stolley.bib}

\end{document}
